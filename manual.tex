\documentclass{article}
\usepackage{amsmath}

\newcommand{\BigO}[1]{\ensuremath{\operatorname{O}\bigl(#1\bigr)}}
\begin{document}
	\begin{center}
	{\Huge \bf Swarm Simulation} \\
	{\large Daniel Lovasko} \\[1\baselineskip]
	\end{center}
\section*{Introduction}
Swarm simulation written in C++ with usage of graphics libraries SDL and
OpenGL, while utilising the computing power of OpenCL. The swarm is composed of
its members, {\it mosquitoes}, and contains predators, {\it dragonflies}.
This document aims to describe the algorithms and ideas used in the program.

\section{Mosquito Rules} This section describes the behavior of the swarm
members - mosquitoes. Altogether, they obey 5 basic rules, each with different
weights. Each rule computes a final velocity. Afterwards, all rules are
combined with simple component-wise vector addition and added to the velocity
of the mosquito in question.

\subsection{Rule 1 - Perceived Centre of Mass}
Each mosquito understands the power of the group and therefore tries to stick
with the swarm. It analyzes the positions of {\it all other} mosquitoes and
computes the {\it arithmetic mean} of these positions and tries to head to this
location.
\subsection{Rule 2 - Personal Space}
Each mosquito needs some personal space to breathe and live happily. By
searching the perimeter of the desired personal circle, it chooses to move away
from all other members of the swarm in this circle.
\subsection{Rule 3 - Velocity Matching}
Trying to keep with the swarm is vital to using the "public knowledge" and each
mosquito tries to mimic the velocity of other swarm members. For example, the
mosquito in front sees a predator and the rule 5 overpowers all other, he tries
to escape in completely other direction. Other mosquitoes, while combining this
rule with the rule 5 of their own, get persuaded to change their direction too.
\subsection{Rule 4 - Border Force}
In order to keep the game of life in some bounded space, each border possesses
a anti-force: top border pushes the mosquito down, left border right, bottom up
and right to the left. As a consequence, nobody is able to escape the pond.
\subsection{Rule 5 - Fear of the Predator}
The motivation behind this rule needs no explanation. Simply, the closer the
predator is, the higher the motivation to escape.

\section{Dragonfly Rules} This section describes the behavior of the predator -
the dragonfly. There is only one rule used, but many possible variants or
improvements exists and are described too.
\subsection{Used Rule - Chase the Closest}
Choose the closest mosquito from the swarm and try to catch it. This ensures
the never-ending dynamic of the swarm, since it motivates the flow at each
time.
\subsection{Possible Rule - Chase the Centre of Mass}
Analyze where the centre of the mass lays and try to move there. This,
unfortunately, seems as only a short-time strategy. The swarm creates a circle
around the predator and does not move (or moves only slightly from edge to
edge) without releasing the predator. Even though he is happy being in the
centre of the mass, he is unable to catch anybody.

\subsection{Possible Rule - Chase the Slowest Member of the Swarm}
Strategy that may result in success - only the predator needs to be faster than
the motivation of the swarm member to escape.

\end{document}

